% --- LaTeX Presentation Template - S. Venkatraman ---

% --- Set document class ---

% Remove "handout" when presenting to include pauses
\documentclass[dvipsnames, handout]{beamer}
\usetheme{default}

% Make content that is hidden by pauses "transparent"
\setbeamercovered{transparent}

% Import package for indicator function symbol
\usepackage{dsfont}

% --- Slide layout settings ---

% Set line spacing
\renewcommand{\baselinestretch}{1.15}

% Set left and right text margins
\setbeamersize{text margin left=12mm, text margin right=12mm}

% Add slide numbers in bottom right corner
\setbeamertemplate{footline}[frame number]

% Remove navigation symbols
\setbeamertemplate{navigation symbols}{}

% Allow local line spacing changes
\usepackage{setspace}

% Change itemized list bullets to circles
\setbeamertemplate{itemize item}{$\bullet$}
\setbeamertemplate{itemize subitem}{$\circ$}

% --- Color and font settings ---

\usepackage{xcolor}

% Slide title background color
\definecolor{background}{HTML}{edebe4}

% Slide title text color
\definecolor{titleText}{HTML}{B40404}

% Other possible color schemes

% - Light green/dark green -
%\definecolor{background}{HTML}{e4ede4}
%\definecolor{titleText}{HTML}{2e592f}

% - Light blue/dark blue -
%\definecolor{background}{HTML}{d5d9e8}
%\definecolor{titleText}{HTML}{2d375e}

% - Beige/dark blue -
%\definecolor{background}{HTML}{edebe4}
%\definecolor{titleText}{HTML}{2d3375}

% Set colors
\setbeamercolor{frametitle}{bg=background, fg=titleText}
\setbeamercolor{subtitle}{fg=titleText}

% Set font sizes for frame title and subtitle
\setbeamerfont{frametitle}{size=\fontsize{15}{16}}
\setbeamerfont{framesubtitle}{size=\small}

% --- Math/Statistics commands ---

% Add a reference number to a single line of a multi-line equation
% Usage: "\numberthis\label{labelNameHere}" in an align or gather environment
\newcommand\numberthis{\addtocounter{equation}{1}\tag{\theequation}}

% Shortcut for bold text in math mode, e.g. $\b{X}$
\let\b\mathbf

% Shortcut for bold Greek letters, e.g. $\bg{\beta}$
\let\bg\boldsymbol

% Shortcut for calligraphic script, e.g. %\mc{M}$
\let\mc\mathcal

% \mathscr{(letter here)} is sometimes used to denote vector spaces
\usepackage[mathscr]{euscript}

% Convergence: right arrow with optional text on top
% E.g. $\converge[p]$ for converges in probability
\newcommand{\converge}[1][]{\xrightarrow{#1}}

% Weak convergence: harpoon symbol with optional text on top
% E.g. $\wconverge[n\to\infty]$
\newcommand{\wconverge}[1][]{\stackrel{#1}{\rightharpoonup}}

% Equality: equals sign with optional text on top
% E.g. $X \equals[d] Y$ for equality in distribution
\newcommand{\equals}[1][]{\stackrel{\smash{#1}}{=}}

% Normal distribution: arguments are the mean and variance
% E.g. $\normal{\mu}{\sigma}$
\newcommand{\normal}[2]{\mathcal{N}\left(#1,#2\right)}

% Uniform distribution: arguments are the left and right endpoints
% E.g. $\unif{0}{1}$
\newcommand{\unif}[2]{\text{Uniform}(#1,#2)}

% Independent and identically distributed random variables
% E.g. $ X_1,...,X_n \iid \normal{0}{1}$
\newcommand{\iid}{\stackrel{\smash{\text{iid}}}{\sim}}

% Sequences (this shortcut is mostly to reduce finger strain for small hands)
% E.g. to write $\{A_n\}_{n\geq 1}$, do $\bk{A_n}{n\geq 1}$
\newcommand{\bk}[2]{\{#1\}_{#2}}

% Math mode symbols for common sets and spaces. Example usage: $\R$
\newcommand{\R}{\mathbb{R}}	% Real numbers
\newcommand{\C}{\mathbb{C}}	% Complex numbers
\newcommand{\Q}{\mathbb{Q}}	% Rational numbers
\newcommand{\Z}{\mathbb{Z}}	% Integers
\newcommand{\N}{\mathbb{N}}	% Natural numbers
\newcommand{\F}{\mathcal{F}}	% Calligraphic F for a sigma algebra
\newcommand{\El}{\mathcal{L}}	% Calligraphic L, e.g. for L^p spaces

% Math mode symbols for probability
\newcommand{\pr}{\mathbb{P}}	% Probability measure
\newcommand{\E}{\mathbb{E}}	% Expectation, e.g. $\E(X)$
\newcommand{\var}{\text{Var}}	% Variance, e.g. $\var(X)$
\newcommand{\cov}{\text{Cov}}	% Covariance, e.g. $\cov(X,Y)$
\newcommand{\corr}{\text{Corr}}	% Correlation, e.g. $\corr(X,Y)$
\newcommand{\B}{\mathcal{B}}	% Borel sigma-algebra

% Other miscellaneous symbols
\newcommand{\tth}{\text{th}}	% Non-italicized 'th', e.g. $n^\tth$
\newcommand{\Oh}{\mathcal{O}}	% Big-O notation, e.g. $\O(n)$
\newcommand{\1}{\mathds{1}}	% Indicator function, e.g. $\1_A$

% Additional commands for math mode
\DeclareMathOperator*{\argmax}{argmax}	% Argmax, e.g. $\argmax_{x\in[0,1]} f(x)$
\DeclareMathOperator*{\argmin}{argmin}	% Argmin, e.g. $\argmin_{x\in[0,1]} f(x)$
\DeclareMathOperator*{\spann}{Span}	% Span, e.g. $\spann\{X_1,...,X_n\}$
\DeclareMathOperator*{\bias}{Bias}	% Bias, e.g. $\bias(\hat\theta)$
\DeclareMathOperator*{\ran}{ran}		% Range of an operator, e.g. $\ran(T) 
\DeclareMathOperator*{\dv}{d\!}		% Non-italicized 'with respect to', e.g. $\int f(x) \dv x$
\DeclareMathOperator*{\diag}{diag}	% Diagonal of a matrix, e.g. $\diag(M)$
\DeclareMathOperator*{\trace}{trace}	% Trace of a matrix, e.g. $\trace(M)$
\DeclareMathOperator*{\supp}{supp}	% Support of a function, e.g., $\supp(f)$

% --- Presentation begins here ---

\begin{document}

% --- Title slide ---

\title{\color{titleText}{\Large Introduction to \\[.7em]\includegraphics[width=5cm]{Logo}\\ for statistics and probability}}
\author{\textbf{Cornell Statistics Graduate Society}\\Sara Venkatraman}
\date{}

\begin{frame}
\vspace{1.2cm}\titlepage
\end{frame}

% --- Main content ---

% Example slide: use \pause to sequentially unveil content
\begin{frame}{Outline}
\begin{enumerate}
\item What is Mathematica and why should you use it?
\item How to obtain Mathematica
\item Live demo
\item Example of Mathematica in research
\end{enumerate}
\end{frame}

\begin{frame}{What is Mathematica?}
\begin{itemize}
\item Mathematica is a \textbf{symbolic} computing system: can manipulate mathematical expressions, unlike R or Python
\item Uses the Wolfram programming language
\item Example: let $f(x,\theta) = (\theta x^3 + 1)^{-1}$. What is $\partial f/\partial x$?
\end{itemize}
\begin{figure}[H]
\centering\fbox{\includegraphics[width=8cm]{DerivExample}}	
\end{figure}
\end{frame}

\begin{frame}{What is Mathematica?}
For statisticians, I think Mathematica is very useful for:	
\begin{itemize}
\item Verifying tedious calculus: derivatives, integrals, etc.
\item Analytically solving complicated equations
\item Working with random variables {\scriptsize (not realizations of them)}
\end{itemize}\vspace{.5cm}

I think Mathematica is less useful for:
\begin{itemize}
\item Working with and visualizing real data	
\item Larger and potentially open-source software
\end{itemize}
\end{frame}

\begin{frame}{How to get Mathematica}
\begin{itemize}
\item \$35 for one-year license (Aug. 1 - July 31) from Cornell IT; also includes Mathematica Online and WolframAlpha Pro: {\color{RoyalBlue}\url{https://it.cornell.edu/software-licensing/mathematica-licensing}}\\[1em]
\item 15-day free trial: {\color{RoyalBlue}\url{https://www.wolfram.com/mathematica/trial/}}
\end{itemize}
\end{frame}

\begin{frame}
\begin{center}\Large\color{titleText}Live Mathematica demo\end{center}	
\end{frame}

\begin{frame}{Mathematica in my research}

{\small A Bayesian approach to regression:
\begin{itemize}
\item Consider the model $\b Y = \b X\bg\beta + \bg\varepsilon$, with $\bg\varepsilon \sim N(\b 0,\sigma^2 \b I_n)$. 
\item Put a $N(\bg\beta_0,\sigma^2\b V_0)$ prior on $\bg\beta$, where $\b V_0$ is some p.s.d. matrix.
\item What should $\b V_0$ be? Let's say $\b V_0 = g(\b X^T \b X)^{-1}$, for some $g>0$
\item Then the posterior mean of $\bg\beta$ is:
$$\E(\bg\beta|\b Y,\sigma^2) = \frac{1}{1+g}\bg\beta_0 + \frac{g}{1+g}\bg{\hat\beta}$$
where $\bg{\hat\beta}$ is the usual least-squares estimate!
\item \textbf{What I want to know: How do we choose $g$?}
\end{itemize}}
\end{frame}

\begin{frame}{Mathematica in my research (continued)}

{\small We could try solving for $g$ that minimizes sum of squared residuals,\\[.5em] $\|\b Y - \b{\hat Y}\|^2$, where $\b{\hat Y} = \b X(\frac{1}{1+g}\bg\beta_0+\frac{g}{1+g}\bg{\hat\beta}$).\\[1em]

After some algebra, I can write this as:
{\scriptsize\begin{align*}
\text{SSR}(g)=\|\b Y - \b{\hat Y}\|^2 &= a-\frac{2b}{1+g}-\frac{2gc}{1+g}+\frac{d}{(1+g)^2}+\frac{2gb}{(1+g)^2}+\frac{g^2c}{(1+g)^2}
\end{align*}}
where $a=\|\b Y\|^2$, $b=\b Y^T\b Y_0$, $c=\|\bg{\hat\beta}\|^2$, $d=\|\b Y_0\|^2$, $\b Y_0 = \b X\bg\beta_0$.
\begin{figure}[H]
\centering\fbox{\includegraphics[width=10cm]{SSR}}	
\end{figure}
Turns out there are no minimizing values of $g$... ~ :(
}
\end{frame}

\begin{frame}{Mathematica in my research (continued)}
	
{\small Instead let's minimize \textit{Stein's unbiased risk estimate} (SURE), given by:\\[.5em] 
$\|\b Y - \b X\bg{\hat\beta}\|^2 + \frac{2gp\hat\sigma^2}{1+g}-n\hat\sigma^2$.\\[1em]

After some algebra, I can write this as:
{\scriptsize\begin{align*}
\text{SURE}(g)= a-\frac{2b}{1+g}-\frac{2gc}{1+g}+\frac{d}{(1+g)^2}+\frac{2gb}{(1+g)^2}+\frac{g^2c}{(1+g)^2}+\frac{2gp\hat\sigma^2}{1+g}-n\hat\sigma^2
\end{align*}}\\[-1em]
\begin{figure}[H]
\centering\fbox{\includegraphics[width=10cm]{SURE}}	
\end{figure}
Now we have a solution; let's check that it's indeed a minimizer.}
\end{frame}

\begin{frame}{Mathematica in my research (continued)}

\begin{figure}[H]
\centering\fbox{\includegraphics[width=10cm]{SURE2}}	
\end{figure}
Now all I need to do is prove that this quantity is positive.
\end{frame}


% --- Thank you slide ---

\begin{frame}
\begin{center}
{\Large\color{titleText} Thank you for listening!\\[2em]}

All the things Mathematica can do are\\ documented here, with examples:\\ {\color{RoyalBlue}\url{https://reference.wolfram.com/language/}}
\end{center}
\end{frame}

% --- Presentation ends here ---

\end{document}
